\section{函数与极限}

本章涉及了高等数学中基本的玩具.

\subsection{函数的初等性态}{研究函数性质的几个角度}
\begin{itemize}
\item 奇偶性
\item 周期性
\item 单调性
\item 有界性
\end{itemize}

\subsection{一些新玩具}
\subsubsection{等式}
\(a^n-b^n=(a-b)(a^{n-1}+a^{n-2}b+a^{n-3}b^2+\cdots+a^2b^{n-3}+ab^{n-2}+b^{n-1})\)

\subsubsection{不等式}
(Cauchy-Schwartz不等式) 对任意\(a_i,b_i\in\mathbf{R}\), 有
\[\left(\sum_{i=1}^na_ib_i\right)^2\le\sum_{i=1}^na_i^2\sum_{i=1}^nb_i^2.\]

\subsubsection{三角函数}
\begin{description}
\item[余切] \(\displaystyle y=\cot x=\frac{\cos x}{\sin x},\{x\mid x\ne n\pi,n=0,\pm 1,\pm 2,\cdots\}\)
\item[正割] \(\displaystyle y=\sec x=\frac{1}{\cos x},\{x\mid x\ne n\pi+\frac{\pi}{2},n=0,\pm 1,\pm 2,\cdots\}\)
\item[余割] \(\displaystyle y=\csc x=\frac{1}{\sin x},\{x\mid x\ne n\pi,n=0,\pm 1,\pm 2,\cdots\}\)
\end{description}

\subsubsection{反三角函数}

\subsubsection{双曲函数}
\paragraph{定义}
\begin{description}
\item[双曲正弦函数] \(\displaystyle \sh x=\frac{e^x-e^{-x}}{2},D=\mathbf{R}\)
\item[双曲余弦函数] \(\displaystyle \ch x=\frac{e^x+e^{-x}}{2},D=\mathbf{R}\)
\item[双曲正切函数] \(\displaystyle \th x=\frac{\sh x}{\ch x}=\frac{e^x-e^{-x}}{e^x+e^{-x}},D=\mathbf{R}\)
\item[双曲余切函数] \(\displaystyle \cth x=\frac{\ch x}{\sh x}=\frac{e^x+e^{-x}}{e^x-e^{-x}},D=\mathbf{R}\)
\item[反双曲正弦] \(\arcsh x=\ln (x+\sqrt{x^2+1}),D=\mathbf{R}\)
\item[反双曲余弦] \(\arcch x=\ln (x+\sqrt{x^2-1}),D=[1,+\infty)\)
\item[反双曲正切] \(\displaystyle \arcth x=\frac{1}{2}\ln \frac{1+x}{1-x},D=(-1,1)\)
\end{description}

\paragraph{性质}
\begin{itemize}
\item \(\mathrm{ch}^2\,x-\mathrm{sh}^2\,x=1\)
\item \(\sh 2x=2\sh x\ch x\)
\item \(\ch 2x=\mathrm{sh}^2\,x+\mathrm{ch}^2\,x\)
\item \(\sh (x\pm y)=\sh x\ch y\pm \ch x\sh y\)
\item \(\ch (x\pm y)=\ch x\ch y\pm \sh x\sh y\)
\end{itemize}

\subsection{数列的极限}
\subsubsection{定义}
\(\forall\varepsilon>0\), \(\exists N>0\), 当\(n>N\)时, 恒有\(|x_n-a|<\varepsilon\)成立.

\subsubsection{性质}
\begin{description}
\item[极限的唯一性] 若数列\(\{x_n\}\)收敛, 则它的极限唯一.
\item[收敛数列的有界性] 若数列\(\{x_n\}\)收敛, 则\(\{x_n\}\)有界.
\item[收敛数列的保序性] 设有数列\(\{x_n\}\), \(\{y_n\}\), \(\displaystyle\lim_{n\to\infty}x_n=a\), \(\displaystyle\lim_{n\to\infty}y_n=b\), 且自某一项起, 有\(x_n\le y_n\), 则\(a\le b\).
\item[子列的收敛性] 数列\(\{x_n\}\)收敛于\(a\)的充分必要条件是\(\{x_n\}\)的任一子列都收敛, 且都收敛于\(a\).
\end{description}

\subsection{函数的极限}

\subsection{求极限的方法}
\subsubsection{按玩法常用程度排序}
\begin{enumerate}
\item 极限的四则运算, 极限的复合运算
\item 初等函数的连续性, \(\displaystyle\lim_{x\to x_0}f(x)=f(x_0)\)
\item 特殊极限
\item 等价无穷小 (相乘除)
\item 夹逼定理
\item 有界量\(\times\)无穷小\(=\)无穷小
\item 单调有界 (证明)
\item 其它: 分子有理化
\end{enumerate}

\subsection{极限的定义}

\subsubsection{一系列等价无穷小}
\[\displaystyle x\sim\sin x\sim\tan x\sim\arcsin x\sim\arctan x\sim e^x-1\sim\ln(1+x)\;(x\to 0)\]
\[(1+x)^\alpha-1\sim\alpha x\;(x\to 0)\]
\[1-\cos x\sim\frac{x^2}{2}\;(x\to 0)\]

\subsection{函数的连续性与间断点}
连续的定义
